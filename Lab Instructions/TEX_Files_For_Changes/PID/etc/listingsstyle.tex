% =========================================================
% Colorful listings
%
% Needed: \uspackage{listings, xcolor}
% =========================================================

\lstdefinestyle{cpp}{
	language=C++,
	keepspaces=true,
	basicstyle=\ttfamily\footnotesize,
	keywordstyle=\color{blue}\ttfamily,
	stringstyle=\color{red}\ttfamily,
	commentstyle=\itshape\color{gray}\ttfamily,
	morecomment=[l][\color{red}]{\#},
	numbers=left,
	numberstyle={\tiny \color{gray}},% size of the numbers
	frame=single,
	breaklines=true,
	tabsize=4
}
\lstdefinestyle{asm}{
	language=[x86masm]Assembler,
	keepspaces=true,
	basicstyle=\ttfamily\footnotesize,
	keywordstyle=\ttfamily,
	stringstyle=\color{red}\ttfamily,
	commentstyle=\itshape\color{gray}\ttfamily,
	numbers=left,
	numberstyle={\tiny \color{gray}},% size of the numbers
	frame=single,
	breaklines=true,
	tabsize=4,
}
\lstdefinestyle{c}{
	language=C,
	keepspaces=true,
	basicstyle=\ttfamily\footnotesize,
	keywordstyle=\ttfamily,
	stringstyle=\color{red}\ttfamily,
	commentstyle=\itshape\color{gray}\ttfamily,
	numbers=left,
	numberstyle={\tiny \color{gray}},% size of the numbers
	frame=single,
	breaklines=true,
	tabsize=4,
}
\lstdefinestyle{shell}{
	language=sh,
	keepspaces=true,
	basicstyle=\ttfamily\footnotesize,
	keywordstyle=\color{red}\ttfamily,
	keywordstyle=[1]{\bfseries\color{black}},
	morekeywords={sudo, mkdir, cp},
	stringstyle=\color{gray}\ttfamily,
	commentstyle=\itshape\color{gray}\ttfamily,
	% morecomment=[l][\color{red}]{\#},
	numbers=left,
	numberstyle={\tiny \color{gray}},% size of the numbers
	frame=single,
	breaklines=true
}
\lstdefinestyle{matlab}{
	language=Matlab,%
	breaklines=true,%
	frame=single,%
	morekeywords={matlab2tikz},
	morekeywords={xlim,ylim,var,alpha,factorial,poissrnd,normpdf,normcdf,audioread},
	basicstyle=\ttfamily\footnotesize,
	keywordstyle=[1]{\bfseries\color{black}},        % MATLAB functions bold and blue
	keywordstyle=[2]{\bfseries\color{black}},         % MATLAB function arguments purple
	keywordstyle=[3]{\color{Blue}\underbar},     % User functions underlined and blue
	tabsize=5,                              % 5 spaces per tab
	identifierstyle=\color{black},%
	stringstyle=\color{purple},
	commentstyle=\color{gray},%
	showstringspaces=false,%without this there will be a symbol in the places where there is a space
	numbers=left,%
	numberstyle={\tiny \color{gray}},% size of the numbers
	numbersep=9pt, % this defines how far the numbers are from the text
	emph=[1]{for,end,break},emphstyle=[1]\color{blue}, %some words to emphasise
}
\lstdefinestyle{matlabInLine}{
	identifierstyle=\color{blue},%
	language=Matlab,%
	basicstyle=\ttfamily\footnotesize,
	stringstyle=\color{purple},
	emph=[1]{for,end,break},emphstyle=[1]\color{blue}, %some words to emphasise,
}

% Deutsche Sonderzeichen in Listing
\lstset{
	frame=singe,
	basicstyle=\footnotesize\ttfamily,
	breaklines=true,
	numbers=left,
	inputencoding=utf8,
	extendedchars=true,
	literate={ä}{{\"a}}1 {ö}{{\"o}}1 {ü}{{\"u}}1,
}

% Linebreaks im \texttt erlauben bei []/.
\renewcommand{\texttt}[1]{%
	\begingroup
	\ttfamily
	\begingroup\lccode`~=`/\lowercase{\endgroup\def~}{/\discretionary{}{}{}}%
	\begingroup\lccode`~=`[\lowercase{\endgroup\def~}{[\discretionary{}{}{}}%
	\begingroup\lccode`~=`.\lowercase{\endgroup\def~}{.\discretionary{}{}{}}%
	\catcode`/=\active\catcode`[=\active\catcode`.=\active
	\scantokens{#1\noexpand}%
	\endgroup
}